\documentclass{./template/template}
% -------------------- Quadros e lista de quadros:
\usepackage{array} % aux. tabelas
\usepackage{float} % floats
\usepackage{caption} % legendas
\usepackage{tocloft} % Sumário/listas (tab./fig.)
% Espaçamento entre linhas (tabelas):
\renewcommand{\arraystretch}{1.25} 
%
% ------------------------------------------------
% Definir um novo tipo de float chamado 'quadro'
\floatstyle{plaintop}
\newfloat{quadro}{!ht}{lop}
\floatname{quadro}{Quadro}
% Configurar a lista de quadros
\newcommand{\listofquadros}{
  \listof{quadro}{Lista de Quadros}
}


\begin{document}

\listofquadros % lista de quadros
\clearpage     % qubra de página

\section*{Exemplo}

% Parágrafo 2 do lipsum:
\lipsum[2] Quadro \ref{quadro2}

% Quadro 1:
\begin{quadro}[!ht]
\caption{Exemplo de quadro.}
\centering
\begin{tabular}{|c|c|c|}
\hline
\textbf{Coluna 1} & \textbf{Coluna 2} & \textbf{Coluna 3} \\
\hline
1 & 2 & 3 \\
\hline
4 & 5 & 6 \\
\hline
7 & 8 & 9 \\
\hline
\end{tabular}
\end{quadro}

% Parágrafo 3 do "lipsum":
\lipsum[3] Conforme ilustrado no quadro \ref{quadro2}, ...

% Quadro 2:
\begin{quadro}[!ht]
\caption{Outro exemplo de quadro.}
\label{quadro2}
\centering
\begin{tabular}{|c|c|c|}
\hline
\textbf{Coluna 1} & \textbf{Coluna 2} & \textbf{Coluna 3} \\
\hline
1 & 2 & 3 \\
\hline
4 & 5 & 6 \\
\hline
7 & 8 & 9 \\
\hline
\end{tabular}
\end{quadro}

\end{document}
